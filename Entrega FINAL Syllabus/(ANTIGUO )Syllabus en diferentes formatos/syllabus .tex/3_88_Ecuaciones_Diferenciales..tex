\documentclass{article}
\usepackage[utf8]{inputenc}
\usepackage[spanish]{babel}
\usepackage{color,graphicx}
\usepackage{array}

\title{{\bf Facultad de Ingeniería} \\ {\small Proyecto Curricular de Ingeniería Electrónica} \\ Syllabus}
\date{30 de junio de 2023}

\begin{document}
\maketitle

\newcommand{\var}[1]{}
\newcommand{\varlist}[1]{}

\noindent\rule[0.8ex]{12.1cm}{0.25pt} \par
\noindent {\sc Espacio Académico:} {\bf Ecuaciones Diferenciales} \\
\noindent Código del Espacio: {\bf 88} \qquad Número de Créditos: {\bf 3} \\
\noindent Área de Formación: {\bf \var{Área de Formación}} \\

\noindent\rule[0.8ex]{12.1cm}{0.25pt} \par
\noindent {\sc Docente:} {\bf \var{Docente}} \\

\noindent\rule[0.8ex]{12.1cm}{0.25pt} \par
\noindent {\sc Tiempo asignado por método de enseñanza y aprendizaje} \\
\noindent Horas Trabajo Directo: {\bf 4} \\ 
          Horas Trabajo Colaborativo: {\bf 2} \\ 
		  Horas Trabajo Autónomo: {\bf 3} \\

\noindent\rule[0.8ex]{12.1cm}{0.25pt} \par
\noindent {\sc Categoría del Espacio Académico} \\
\begin{tabular}{lclc}
Obligatorio & (X)  & Electivo       & () \\
Básico      & (X)  & Complementario & () \\
Intríseco   & () & Extrínseco     & () \\
\end{tabular} \\

\noindent\rule[0.8ex]{12.1cm}{0.25pt} \par
\noindent {\sc Tipo de Curso} \\
\begin{tabular}{lclc}
Teórico          & (X) & Práctico   & () \\
Teórico-Práctico & (\var{Teórico-Práctico}) & Asistido por TICs & (\var{Asistido por TICs}) \\
\end{tabular} \\

\noindent\rule[0.8ex]{12.1cm}{0.25pt} \par
\noindent {\sc Alternativas Metodológicas} \\
\begin{tabular}{lclclc}
Clase magistral & () & Seminario & () & Seminario-Taller & () \\
Taller & () & Prácticas & () & Proyectos con tutoría & () \\
\end{tabular} \\

\noindent\rule[0.8ex]{12.1cm}{0.25pt} \par
\noindent {\sc Conocimientos previos del curso} \\
\noindent Conocimientos sobre  \\

\noindent\rule[0.8ex]{12.1cm}{0.25pt} \par
\noindent {\sc Competencias y Resultados de Aprendizaje} \\
\begin{tabular}{|l|c|c|p{5.1cm}|} \hline
{\bf Competencias} & {\bf Nivel} & {\bf RA} & {\bf Resultados de aprendizaje} \\ \hline
Básicas & Conocimiento & 1 & Recordar las normas de ortografía generales del idioma. \\
 & & 2 & Listar los propósitos de los signos de puntuación. \\
 & & 3 & Identificar los diferentes tipos de edición: texto normal, matemático, algorítmico.\\
 & Comprensión & 1 & Describir los diferentes tipos de documentos técnicos propios de la ingeniería: reseñas, informes, reportes, ponencias, artículos, manuales, libros. \\
 & & 2 & Reconocer a la lectura como el medio de adquisición de nuevo conocimiento \\ \hline
Comunicativas & Aplicación & 1 & Construir una presentación oral con ayuda de material audio-visual. \\
 & & 2 & Elaborar ponencias o artículos a través de plantillas técnicas. \\ 
 & Análisis & 1 & Examinar los diferentes conceptos estudiados en diferentes tipos de documentos. \\
 & & 2 & Distinguir la relación entre los fenómenos de la naturaleza y los modelos que los representan. \\ \hline
\end{tabular} \\

\noindent\rule[0.8ex]{12.1cm}{0.25pt} \par
\noindent {\sc Contenidos y Unidades Temáticas} \\
\begin{tabular}{cp{11cm}}
  \textbullet & Contenido 1 \\
  \textbullet & Contenido 2 \\
  \textbullet & ... \\
  \textbullet & Contenido n \\
\end{tabular} \\
%\begin{tabular}{cp{11cm}}
%  \textbullet & La comunicación oral y escrita como medio de desarrollo de la profesión. \\
%  \textbullet & Tipos de textos técnicos: apuntes, presentaciones, informes, reseñas, ensayos, reportes de investigación, artículos técnicos y científicos. \\
%  \textbullet & La redacción de textos técnicos en ingeniería. \\
%  \textbullet & Edición de textos técnicos: normal y matemático (\LaTeX). \\ 
%  \textbullet & La lectura como medio de adquisición de nuevo conocimiento. \\
%\end{tabular} \\

\noindent\rule[0.8ex]{12.1cm}{0.25pt} \par
\noindent {\sc Enfoque de Aprendizaje y Enseñanza} \\

\noindent\rule[0.8ex]{12.1cm}{0.25pt} \par
\noindent {\sc Plan de Evaluaciones} \\
%\begin{tabular}{|m{5.5cm}|c|c|c|c|c|c|} \hline
{\bf Resultados de Aprendizaje} \\%& \multicolumn{6}{c|}{\bf Result. de aprendizaje asociados} \\
\begin{tabular}{cp{11cm}}
  \textbullet & Resultado 1 \\
  \textbullet & Resultado 2 \\
  \textbullet & ... \\
  \textbullet & Resultado n \\
\end{tabular} \\
%{\bf a ser evaluados} & \multicolumn{6}{c|}{(T: Teórico / P: Práctico)} \\ \hline
%Normas de ortografía generales & \hspace{1.2em} & \hspace{1.2em} & \hspace{1.2em} & \hspace{1.2em} & \hspace{1.2em} & \hspace{1.2em} \\ \hline
%Uso de los signos de puntuación & & & & & & \\ \hline
%Edición de reseña de ensayo o libro & & & & & & \\ \hline
%Presentación oral de tema leído & & & & & & \\ \hline
%Resolver problema y escribir artículo con la solución para publicar & & & & & & \\ \hline
%\end{tabular} \\

\noindent\rule[0.8ex]{12.1cm}{0.25pt} \par
\noindent {\sc Materiales de Estudio} \\
\begin{tabular}{cp{11cm}}
  \textbullet & [1] Álvaro Andrés Hamburguer Fernández, {\it Escribir para objetivar el saber: Cómo producir artículos, libros, reseñas, textos y ensayos}. Segunda Edición, Bogotá, D.C., Universidad de La Salle, 2017. \\
  \textbullet & [2] William Ángel Salazar Pulido, {\it Alta Redacción: Informes científicos, académicos, técnicos y administrativos}. Novena Edición, NET Educativa, Bogotá, D.C., 2009. \\
  \textbullet & [3] Rodrigo de Castro Korgi, {\it El universo \LaTeX}. Segunda Edición, Bogotá: Universidad Nacional de Colombia, 2003. \\
  \textbullet & [4] Hilary Glasman-Deal, {\it Science Research Writing: For Non-Native Speakers of English}. Imperial College Press, London, 2010. \\
  \textbullet & [5] Estanislao Zuleta, {\it Elogio de la dificultad y Otros ensayos}. Quinta Edición, Fundación Estanislao Zuleta, Cali, 2001. \\
  \textbullet & [6] Estanislao Zuleta, {\it Arte y Filosofía}. Segunda Edición, Hombre Nuevo Editores, Fundación Estanislao Zuleta, Medellín, 2001. \\
  \textbullet & [7] Academias de la Lengua Española, {\it Ortografía de la Lengua Española}. Editorial Espasa Calpe, S.A., Madrid, 1999 \\
  \textbullet & [8] William Ospina, {\it La lámpara maravillosa: Cuatro ensayos sobre la educación y un elogio de la lectura}. Random House Mondadori, S.A.S., Bogotá, 2013. \\
  \textbullet & [9] Peter Watson, {\it Convergencias: El orden subyacente en el corazón de la ciencia}. Editorial Planeta S.A., Bogotá, 2017. \\
  \textbullet & [10] Nils J. Nilsson, {\it Para una comprensión de las creencias}. Fondo de Cultura Económica, México, 2019. \\
\end{tabular} \\
% \begin{tabular}{|c|m{11cm}|} \hline
% [1] & Álvaro Andrés Hamburguer Fernández, { Escribir para objetivar el saber: Cómo producir artículos, libros, reseñas, textos y ensayos}. Segunda Edición, Bogotá, D.C., Universidad de La Salle, 2017. \\ \hline
% [2] & William Ángel Salazar Pulido, {\it Alta Redacción: Informes científicos, académicos, técnicos y administrativos}. Novena Edición, NET Educativa, Bogotá, D.C., 2009. \\ \hline
% [3] & Rodrigo de Castro Korgi, {\it El universo \LaTeX}. Segunda Edición, Bogotá: Universidad Nacional de Colombia, 2003. \\ \hline
% [4] & Hilary Glasman-Deal, {\it Science Research Writing: For Non-Native Speakers of English}. Imperial College Press, London, 2010. \\ \hline
% [5] & Estanislao Zuleta, {\it Elogio de la dificultad y Otros ensayos}. Quinta Edición, Fundación Estanislao Zuleta, Cali, 2001. \\ \hline
% [6] & Estanislao Zuleta, {\it Arte y Filosofía}. Segunda Edición, Hombre Nuevo Editores, Fundación Estanislao Zuleta, Medellín, 2001. \\ \hline
% [7] & Academias de la Lengua Española, {\it Ortografía de la Lengua Española}. Editorial Espasa Calpe, S.A., Madrid, 1999 \\ \hline
% [8] & William Ospina, {\it La lámpara maravillosa: Cuatro ensayos sobre la educación y un elogio de la lectura}. Random House Mondadori, S.A.S., Bogotá, 2013. \\ \hline
% [9] & Peter Watson, {\it Convergencias: El orden subyacente en el corazón de la ciencia}. Editorial Planeta S.A., Bogotá, 2017. \\ \hline
% [10] & Nils J. Nilsson, {\it Para una comprensión de las creencias}. Fondo de Cultura Económica, México, 2019. \\ \hline
% \end{tabular}




\end{document}
